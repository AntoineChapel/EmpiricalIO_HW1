\documentclass[12pt]{article}
\usepackage[utf8]{inputenc}
\usepackage{optidef}
\usepackage{graphicx}
\usepackage{amsmath}
\usepackage{amssymb}
\usepackage{amsfonts}
\usepackage{xcolor}
\usepackage{unicode-math}
\usepackage{hyperref}
\usepackage{setspace}
\usepackage{float}
\usepackage{amsthm}
\usepackage{chngcntr}
\usepackage{eurosym}
\usepackage{caption}
\usepackage{subcaption}
\usepackage{enumitem}
\usepackage{dirtytalk}
\usepackage{enumitem}
\usepackage{tikz}
\usepackage{pgfplots}
\usepgfplotslibrary{fillbetween}
\pgfplotsset{compat = newest}




\usepackage{titling}
\newcommand{\subtitle}[1]{%
  \posttitle{%
    \par\end{center}
    \begin{center}\large#1\end{center}
    \vskip0.5em}%
}


\theoremstyle{plain}
\newtheorem{assumption}{Assumption}
\newtheorem{theorem}{Theorem}[section]
\newtheorem{lemma}[theorem]{Lemma}
\newtheorem{proposition}{Proposition}
\newtheorem{corollary}{Corollary}
\newtheorem{definition}{Definition}


\let\origtheassumption\theassumption

\DeclareMathOperator{\Tr}{Tr}
\DeclareMathOperator{\E}{E}
\DeclareMathOperator{\Var}{Var}

\usepackage[a4paper, total={6in, 10in}]{geometry}

\onehalfspacing
\title{Empirical IO: Homework 1}
\author{Cameron Scalera, Yoel Feinberg, Antoine Chapel}
\date{\today}

\begin{document}

\maketitle

\section*{Problem 1}
\subsection*{Part (a)}

The individual choice probability for individual $h$ to choose option $j$ at
period $t$ is given by:

\begin{align*}
  P_{hjt} &= \frac{\exp(\beta_j^h + \eta^h x_{jt}^h)}{1 + \sum_{j=1}^J \exp(\beta_j^h + \eta^h x_{jt}^h)} \\
\end{align*}

For each individual $h$, we observe a sequence of choices $y_{ht} = (y_{h1},
..., y_{hT})$. The likelihood function for an individual's sequence of choices
is given by:

The individual-level log-likelihood function is given by:
\begin{align*}
  \mathcal{L}_h(\Theta^h) &= \log\big(L_h(\Theta^h)\big) = \sum_{t=1}^T \sum_{j \in J} y_{jt}^h \log(P_{hjt}) \\
  &= \sum_{t=1}^T \sum_{j=1}^J  y_{jt}^h \log\left(\frac{\exp(\beta_j^h + \eta^h x_{jt}^h)}{1 + \sum_{k=1}^J \exp(\beta_k^h + \eta^h x_{kt}^h)}\right) \\
\end{align*}

Where $y_{jt}^h$ is to be correctly interpreted as an indicator for which
alternative individual $h$ has chosen. Given that the choice set does not change
over time, the denominator of the probability remains a simple expression.

The score function for individual $h$ is given by:


\begin{align*}
  \nabla_{\Theta^h} \mathcal{L}_h(\Theta^h) &= \begin{bmatrix}
    \frac{\partial \mathcal{L}_h(\Theta^h)}{\partial \beta_1^h} \\
    \vdots \\
    \frac{\partial \mathcal{L}_h(\Theta^h)}{\partial \beta_J^h} \\
    \frac{\partial \mathcal{L}_h(\Theta^h)}{\partial \eta^h} \\
   \end{bmatrix}
\end{align*}

\begin{align*}
  \frac{\partial \mathcal{L}}{\partial \beta_k^h} &= \sum_j \sum_t y_{jt}^h \frac{\partial}{\partial \beta_k^h} \log(P_{hjt}) \\
\end{align*}

For easy notation, I denote $v_k = \beta_k^h + \eta^h x_{kt}^h$


\begin{align*}
  \frac{\partial P_{hjt}}{\partial \beta_k^h} &= \frac{1 + \sum_k e^{v_k}}{e^{v_j}} \frac{e^{v_j} (1+\sum_k e^{v_k}) - e^{v_j}e^{v_j}}{(1+\sum_k e^{v_k})^2} \\
  &= \frac{1+\sum_k e^{v_k} - e^{v_j}}{1 + \sum_k e^{v_k}} \\
  &= (1-P_{hjt})
\end{align*}


\begin{align*}
  \frac{\partial P_{hjt}}{\partial \beta_j^h} &= \frac{1 + \sum_k e^{v_k}}{e^{v_j}} \frac{e^{v_j} (1+\sum_k e^{v_k}) - e^{v_j}e^{v_j}}{(1+\sum_k e^{v_k})^2} \\
  &= (1-P_{hjt}) x_{jt}^h
\end{align*}




\end{document}
